\begin{問題}{事象と確率} % 質量・長さ・時間(電荷)
サイコロを$4$つ投げたとき,次の確率を求めよ.
\\
(1) 出る目がすべて$1$である確率\\
(2) 出る目がすべて同じである確率\\
(3) ある$2$つのサイコロにおいて出る目が異なる確率\\
(4) どのサイコロも異なる目が出る確率
\end{問題}

\begin{解説}
サイコロを投げるような同様のことを繰り返すことが可能な行為を試行といい,その結果ある目が出るというような事柄を事象という。
特に,事象としてもうこれ以上分けることができない事象を根元事象という。例えば,サイコロの例における「偶数の目が出る」という事象は
「$2$の目が出る」,「$4$の目が出る」,「$6$の目が出る」という事象に分けて考えることができるので根元事象ではないが,
「$3$の目が出る」という事象は$3$の目が出る事象以外の場合に分けて考えられないので,根元事象である。
対象とするすべての根元事象からなる事象の集合を全事象と呼び$U$で表すことにする。
また,事象$A$に対して,事象$A$の場合の数を$n(A)$で表すことにする。全事象$U$に対して,そのすべての根元事象が同様に確からしく起こるとき,
事象$A$の起こる確率$P(A)$は$P(A)=\frac{n(A)}{n(U)}$として定義される。

決して起こらない事象を空事象と呼び,$\emptyset$で表す。定義により,$n(\emptyset)=0$であり,$A=\emptyset$ならば$P(A)=\frac{n(\emptyset)}{n(U)}=0$である。
また,$A=U$ならば$P(A)=\frac{n(U)}{n(U)}=1$である。したがって,一般に確率$P(A)$において$0\leq P(A)\leq 1$が成り立つ。
事象$A$に対して,$A$が起こらないという事象を$A$の余事象といい,$\overline{A}$で表す。定義により,$P(U)=P(A)+P(\overline{A})=1$が成り立つ。

$2$つの事象$A,B$に対して,その和集合と積集合によって定まる事象をそれぞれ和事象,積事象といい,$A\cup B$,$A\cap B$で表す。
$A\cap B=\emptyset$であるとき,$A$と$B$は排反であるという。一般に,$P(A\cup B)=P(A)+P(B)-P(A\cap B)$が成り立ち,$A$と$B$が排反ならば$P(A\cup B)=P(A)+P(B)$が成り立つ。
\end{解説}
\begin{解答}

以下では,対象となる$4$つのサイコロをそれぞれ$a,b,c,d$とおき,これらのサイコロを投げたときに出る目をそれぞれ$f(a),f(b),f(c),f(d)$とおき,出た目の結果を
$(f(a),f(b),f(c),f(d))$で表すことにする。

\VS{1}
\noindent
(1) 出る目がすべて$1$であるという事象を$A$とおくと,事象$A$は$(1,1,1,1)$となる根元事象である。したがって,$n(A)=1$が成り立つ。
一方,この場合の全事象$U$における場合の数$n(U)$は,$(f(a),f(b),f(c),f(d))$で表現できるすべての場合の数$6^4=1296$である。
よって,$P(A)=\frac{1}{1296}$

\noindent
(2) 出る目がすべて同じであるという事象を$B$とおくと,事象$B$の起こり得るパターンは\\
$(1,1,1,1), (2,2,2,2), (3,3,3,3), (4,4,4,4), (5,5,5,5), (6,6,6,6)$の6通り
なので,\linebreak
$n(B)=6$である。全事象$U$における$n(U)$は前問で求めたように$n(U)=1296$なので,$P(B)=\frac{6}{1296}=\frac{1}{216}$

\noindent
(3) ある$2$つのサイコロにおいて出る目が異なるという事象を$C$とおくとき,事象$C$は出る目がすべて同じであるという事象$B$の余事象である。
したがって,前問の結果より,$P(C)=1-P(B)=1-\frac{1}{216}=\frac{215}{216}$ 

\noindent
(4) どのサイコロも異なる目が出るという事象を$D$とおく。場合の数$n(D)$については次のように注意深く求める必要がある。まず,$4$つのサイコロ
を区別することを考えず,出る目のみに着目してどの目も異なるパターンの総数を求めよう。これは$1$から$6$の数字から$4$つの数字を取り出す
組合せの数${}_6 \mathrm{C}_4=\frac{6!}{4!2!}=15$である。一方,どの目も異なる$4$つの数字のひと組に対して,$4$つのサイコロ$a,b,c,d$において
その組の出る目を実現するパターンの総数を考えよう。ここでは,話を分かりやすくするために,一つの具体的な数字の組合せを例えば$1,2,3,4$とおいて
そのパターン数を数え上げてみると,$a,b,c,d$の出る目が$1,2,3,4$であるパターンの総数は$1,2,3,4$を並べ替える順列の数${}_4 \mathrm{P}_4=4!=24$に等しい。
従って,$n(D)=15\times 24=360$が成り立つ。よって,$P(D)=\frac{360}{1296}=\frac{5}{18}$
\end{解答}
